\chapter{Prosjektorganisering}
\section{Utviklingsmodell}
Oppdragsgiver uttrykte et ønske om tett oppfølging og innsikt underveis i prosjektet og god dokumentasjon for fremtidig vedlikeholdsarbeid. Det var også mulighet for endring eller tillegg av funksjoner i programvare, så en smidig tilnærming var naturlig. \\ \\
På grunn av kravet til dokumentasjon tenkte vi raskt på Scrum som en god løsning og gruppen var allerede godt kjent med denne modellen. Vi undersøkte en del andre utviklingsmodeller, der Kanban og RUP kom frem som gode alternativ. Kanban er litt for løst strukturert for såpass uerfarne utviklere, siden det ikke er noen tidskrav for når hver funksjonalitet skal være ferdig. RUP er svært komplekst og ville ført til mye lengre planleggingstid om vi skal implementere den best mulig, siden vi ikke har noen erfaring med denne modellen.\\ \\ 
Under undersøkelsene kom vi over flere element fra andre utviklingsmodeller som vil være fordelaktige å implementere i vårt prosjekt. FDD har gode rutiner for å identifisere funksjonaliteter som legges til i backlogen2 og XP har mange gode verktøy for å standarisere koden og vi vil også implementere parprograming i størst mulig grad. Hvis en av oss ikke er i stand til å møte, eller en part jobber mer enn planlagt, vil vi gå gjennom endringene så fort vi har mulighet til å møtest. \\ \\
I implementasjonen vår av scrum vil Erland Ørbekk og Nils Kubberud fra Nammo ha rollen som Product Owners og Martin vil ha rollen som Scrum Master. Fordi vi ønsker å være selvorganiserende og lære mest mulig om estimering, vil Scrum Master ta på seg en del av ansvaret til Product Owners. Dersom Product Owners har noe spesifikt som de ønsker å prioritere vil det bli prioritert, men de har uttrykt et ønske om en mer veiledende enn bestemmende rolle i prosjektet. \\ \\
Vi valgte å ikke bruke alle elementene fra Scrum, vi valgte i stedet å bare plukke ut elementer som passet oss best. Elementer som daglige møter ble ikke holdt fordi vi hadde god kontroll over hva den andre jobbet med. Vi jobbet tett sammen som gjorde at vi hadde kontroll over hvor mye som hadde blitt gjort. Vi følte derfor at det ble veldig kunstig å ha et møte der vi diskutere progresjon, ettersom vi hadde full kontroll over motpartens arbeid.\\ \\
Vi brukte sprint elementene fra Scrum, vi satte opp møter med Nammo på slutten av hver sprint hvor vi forklarte hva som hadde blitt gjort denne sprinten. Vi klarte ikke å holde sprintene som vi hadde ønsket. Vi hadde problemer med feil estimering som gjorde at vi forandret sprintene. Vi jobbet videre med oppgavene vi hadde blitt utdelt, men oppgavene tok lengre tid enn estimert, det endte med at vi gikk bort fra sprinten.\\ \\
Vi brukte planing poker, men vi merket tidlig at vi fikk veldig lite fra tiden vi brukte. Vi klarte ikke estimere oppgavene,  og følte at hele prosessen var bortkastet tid vi heller skulle brukt på utvikling av programvare.\\ \\
Vi var veldig fornøyd med Srcum sin arbeidsmetode som jobbet på grunnleggende funksjonalitet som så bygges på videre, noe som er et kjerneelement i utviklingsmodellen.\\ \\


\section{Møter}

Møtene med Nammo valgte vi å sette opp litt etter vi hadde behov for dem, men vi prøvde å ta en tur til kontorene hver andre uke. dette ble satt opp litt etter behov, når vi hadde noen spørsmål som ikke lett kunne besvares på mail. Møtene ble i tillegg satt opp ettersom vi hadde noe å vise frem, en demo versjon for eksempel. Vi viste aldri Nammo noen direkt kode, men vi forklarte hva som hadde blitt gjort teoretisk så de kunne på bedre innsikt i hva som hadde blitt gjort. Det var lett å forklare teorien bak programmet ettersom Nils og Erland har en enorm mengde kunnskap i feltet sitt i tillegg gode kunnskaper i geometri og matematikk.\\
\\Et typisk møte på kontorene til Nammo på Raufoss, varierer på hvilket stadium i prosessen vi er. tidlig i prosessen brukte vi mye tid på hvilket verktøy vi skulle bruke, og hvilket som jobbet godt sammen. Når vi kommet lengre ut i utviklingsfasen, brukte vi møtene på spørsmål og uklarheter. Når vi hadde fått svar på det vi lurte på, forklarte vi hva som hadde blitt gjort og eventuelt vist en liten demo av hvordan ting virker. Dette ga Nammo en god oversikt over hvordan jobbingen gikk. Det var de fornøyd med og var mer interessert i noe som var gjort bygd fra bunnen så det kunne videreutvikles.

\subsection{Møte med veileder}
vi satt opp faste møter med Ivar hver tirsdag kl 13:00, disse møtene har hjulpet oss med å holde oss på rett spor. Møtene var avslappet og det var lett å komme med spørsmål og teorier. vi ble ofte sittende å tenke høyt på de største problemene. VI fikk mye ut av møtene, de var i tillegg lette å avlyse hvis det ikke var behov for dem eller det vare noe som krasjet. Etter å ha avlyst noen møter for bare å fortsette å jobbe når vi ikke hadde spørsmål. fant vi ut at det ikke var så lurt ettersom at vi hadde feiltolket oppgaven, merket vi at det er nyttig å ha møte bare for å passe på at vi holder oss på riktig spor.

\subsection{Gruppe møter} 
Vi avtalte å møte på skolen hver dag å jobbe, ettersom vi bare er to studenter på gruppen var det lett å møtes og jobbe sammen. vi jobbet tett i starten, med mye pairprogramming. Dette var nyttig, å jobbe sammen når vi begynte på et prosjekt som var ukjent for oss med et nytt programmeringsspråk. Når vi kom lengre ut i utviklingsprosessen ble vi jobbene mer hver for oss.

