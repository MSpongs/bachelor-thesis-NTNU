\chapter{Møte Referat}

\section{Indroduksjonsmøte med Nammo - 13.01.2016}

\subsection*{Tilstede:}
\begin{multicols}{2}
\begin{itemize}
    \item Jon Anders Sylvarnes
    \item Martin Holltrø Spongsveen
    \item Nils Kubberud
\end{itemize}{}
\end{multicols}
\subsection*{\large Varihet: \enspace45 min}
\textbf{\Large Sammendrag}

\subsection*{Introduksjon}
Vi fikk en rask introduksjon inne fagfeltet, som ga oss den grunnleggene informasjonen vi trengte. Vi disukuterte oppgaven videre og kom til en god forståelse av hva de ønsket ut av programmvaren og hva vi mente var realistisk prosjekt.

\subsection*{Utviklingsprosess}
Vi spurte om Nammo hadde noe spessielle ønsker eller krav på hvordan vi skulle jobbe med prosjektet. Vi diskuterte emne og kom frem til at vi skulle velge prosessen som passet oss best. Kontakt personene ønsket hyppgie tilbakemeldiger, så vi valgte en smidig utvikligsprosess

\subsection*{NDA ( Non-disclosure agreement )}
Med høye sikkerhetsklareringer, spurte vi om det var behov for en NDA under dette prosjektet. Det var usikkert men tvilsomt.

\subsection*{Prosjektavtale}
Vi leverte Nammo med NTNU sin standard prosjektavtale. Den ble sendt videre sendt, og skulle bli underskrevet og sendt tilbake med e-post.

\subsection*{Versjonskontroll}
Vi diskuterte versjonskontroll, om Nammo hadde noen versjonskontroll de brukte lokat eller om vi kunne bruke Bitbucket

\subsection*{Videre møter}
Vi bestemte at vi skulle ha et møte på Raufoss to ganger i måneden, i tillegg til ukentlig status rapporter som blir sendt via e-post til oppdragsgiver og veileder.


\pagebreak

\section{Møte med Nammo - 06.02.2016}

\subsection*{Tilstede:}
\begin{multicols}{2}
\begin{itemize}
    \item Jon Anders Sylvarnes
    \item Martin Holltrø Spongsveen
    \item Nils Kubberud
\end{itemize}{}
\end{multicols}
\subsection*{\large Varihet: \enspace45 min}
\textbf{\Large Sammendrag}

\subsection*{Introduskjon}

Vi startet møte med Nammo med å fortelle hvordan arbeidet har gått så langt å hva vi jobber med. vi hadde en demonstrasjons av arbeidsmiljøet vi bruker og hvilket språk vi har valgt å bruke. vi viste et eksempel på en 3D modell, så de fikk et innblikk i hvordan et ferdig resultatet kunne bli. etter å ha diskutert de forskjellige utfordringene som står fremfor oss kom vi til en enighet at vi skulle ta et skritt tilbake. vi skal ta for oss det mest grunnleggende funksjonaliteten ettersom vi mener det er kritisk å ha en god algoritme i bunn vi kan bygge videre på. Nammo hadde tidligere jobbet med et prosjekt men hadde store problemer med algoritmen sin som slet med utvendige og innvendige hjørner. spesielt utvendige hjørner som kollapser 