\chapter{Møte Referat}

\section{Indroduksjonsmøte med Nammo - 13.01.2017}

\subsection*{Tilstede:}
\begin{multicols}{2}
\begin{itemize}
    \item Jon Anders Sylvarnes
    \item Martin Holltrø Spongsveen
    \item Nils Kubberud
\end{itemize}{}
\end{multicols}
\subsection*{\large Varihet: \enspace45 min}
\textbf{\Large Sammendrag}

\subsection*{Introduksjon}
Vi fikk en rask introduksjon inne fagfeltet, som ga oss den grunnleggene informasjonen vi trengte. Vi disukuterte oppgaven videre og kom til en god forståelse av hva de ønsket ut av programmvaren og hva vi mente var realistisk prosjekt.

\subsection*{Utviklingsprosess}
Vi spurte om Nammo hadde noe spessielle ønsker eller krav på hvordan vi skulle jobbe med prosjektet. Vi diskuterte emne og kom frem til at vi skulle velge prosessen som passet oss best. Kontakt personene ønsket hyppgie tilbakemeldiger, så vi valgte en smidig utvikligsprosess

\subsection*{NDA ( Non-disclosure agreement )}
Med høye sikkerhetsklareringer, spurte vi om det var behov for en NDA under dette prosjektet. Det var usikkert men tvilsomt.

\subsection*{Prosjektavtale}
Vi leverte Nammo med NTNU sin standard prosjektavtale. Den ble sendt videre sendt, og skulle bli underskrevet og sendt tilbake med e-post.

\subsection*{Versjonskontroll}
Vi diskuterte versjonskontroll, om Nammo hadde noen versjonskontroll de brukte lokat eller om vi kunne bruke Bitbucket

\subsection*{Videre møter}
Vi bestemte at vi skulle ha et møte på Raufoss to ganger i måneden, i tillegg til ukentlig status rapporter som blir sendt via e-post til oppdragsgiver og veileder.


\pagebreak

\section{Møte med Nammo - 06.02.2017}

\subsection*{Tilstede:}
\begin{multicols}{2}
\begin{itemize}
    \item Jon Anders Sylvarnes
    \item Martin Holltrø Spongsveen
    \item Nils Kubberud
\end{itemize}{}
\end{multicols}
\subsection*{\large Varihet: \enspace45 min}
\textbf{\Large Sammendrag}\\ \\
Vi startet møte med Nammo med å fortelle hvordan arbeidet har gått så langt å hva vi jobber med. vi hadde en demonstrasjons av arbeidsmiljøet vi bruker og hvilket språk vi har valgt å bruke. vi viste et eksempel på en 3D modell, så de fikk et innblikk i hvordan et ferdig resultatet kunne bli. etter å ha diskutert de forskjellige utfordringene som står fremfor oss kom vi til en enighet at vi skulle ta et skritt tilbake. vi skal ta for oss det mest grunnleggende funksjonaliteten ettersom vi mener det er kritisk å ha en god algoritme i bunn vi kan bygge videre på. Nammo hadde tidligere jobbet med et prosjekt men hadde store problemer med algoritmen sin som slet med utvendige og innvendige hjørner. spesielt utvendige hjørner som kollapser 


\pagebreak

\section{Møte med Nammo - 01.03.2017}

\subsection*{Tilstede:}
\begin{multicols}{2}
\begin{itemize}
    \item Jon Anders Sylvarnes
    \item Martin Holltrø Spongsveen
    \item Nils Kubberud
\end{itemize}{}
\end{multicols}
\subsection*{\large Varihet: \enspace30 min}
\textbf{\Large Sammendrag}\\ \\
Vi kom med en prototype på vår 2D versjon så Nils kunne se hvor langt vi hadde kommet og Han kunne få et innblikk i hvordan sluttproduktet kan bli. Vi hadde møtt på noen problemer med indre hjørner. måten vi hadde løst oppgaven på virket ikke helt som den skulle. vi endte med en skarpere vinkel for hvert steg. vi var usikker på hvordan hjørnet ville utvikle seg gjennom brenntiden. Nlis forklarte oss at et indre hjørne vil beholde vinkelen sin gjennom brennprosessen. vi hadde derfor noe endringer å gjøre på vår forflytningsalgoritme. Vi diskuterte videre hvor nøyaktig 2D versjonen var. Vi hadde ikke fått testet nøyaktigheten, så vi skulle få tilsendt analytiske modeller for den ene stjerneformen vi har jobbet med. Denne skulle gjøre at vi kunne teste nøyaktigheten på vår 2D modell  


\pagebreak

\section{Møte med Nammo - 28.03.2017}

\subsection*{Tilstede:}
\begin{multicols}{2}
\begin{itemize}
    \item Jon Anders Sylvarnes
    \item Martin Holltrø Spongsveen
    \item Nils Kubberud
\end{itemize}{}
\end{multicols}
\subsection*{\large Varihet: \enspace45 min}
\textbf{\Large Sammendrag}\\ \\
Vi startet møte med å fortelle hva som hadde blitt gjort og hva som har skapt problemer. sist møte møtte vi på et problem med indre hjørner, der hjørnene ikke holdt vinkelen sin. Vinkelen ble bare skarpere. For å motvirke dette gikk vi over fra en algoritme som forflyttet seg punkt vis til en algoritme som forflyttet linjestykke. Vi diskuterte problemene som hadde oppstått med denne overgangen og hvilken bugs som hadde med fulgt. Erland kommenterte på problemer han selv hadde møtt på under sitt forsøk på denne oppgaven og hvordan han hadde prøvd seg på 3D versjonen. Etter Vi var ferdig med hva vi hadde jobbet med, var de interessert i hvordan vi jobbet. Vi fortalte at mesteparten av jobbingen skjer sammen.



