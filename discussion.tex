\chapter{Diskusjon}
\label{chap:discussion}

\section{Resultat}
Vi startet på et prosjekt med et klart må Nammo ønsket et verktøy som kunne hjelpe dem estimere brennsyklusen til en faststoffrakett.  Når vi startet prosjektet var det uklart hvor mye vi ville klare å få til. Vi endte opp med en 2D versjon som hadde en godt grunnleggende algoritme som klarte å håndtere indre og yttrehjørner. Vi ble eninge med Nammo å fokusere å den grunnleggende algoritmen å gjøre den så god som mulig. Ettersom tidligere forsøk på oppgaven hadde møtt på problemer når de skulle gå fra en 2D modell til en 3D modell. Nammo ønsket derfor at vi skulle komme opp med gode byggeklosser som kunne utvides senere. Vi ble fortalt tidlig at vi var ikke de første som hadde jobbet med dette prosjektet. Nammo hadde jobbet i noen år til å få dette prosjektet gjennomført, men uten suksess. Dette ga oss noen bekymringer ettersom Nammo mente at dette var en oppgave som burde vært på en master oppgave for en større gruppe. Under planleggingsfasen hadde vi problemer med å legge en konkret plan over utviklingsfasen. Vi hadde ingen erfaring med grafisk  eller 3D programmering i tillegg til dette skulle prosjektet utvikles i Python, et programmeringsspråk ingen av oss hadde brukt før. Dette ga oss en stor utfordring i begynnelsen, vi brukte derfor mye tid i starten av prosjektet til å undersøke å lære mer om plattformen, miljøet og fagfelte vi skulle jobbe med. Når vi ser tilbake på prosjektet kan vi se at det har gått mye tid til undersøkelse av ukjent språk og mindre effektivt arbeid på grunn av vår mangel på erfaring med platformen vi jobbet på. I tillegg til alle tiden som ble brukt på undersøkelse føler vi oss trygge på at det var det riktige valget. Vi har sett hvor effektivt Python er til å lage prototyper og tiden som kan bli spart på valget vi har tatt. Hadde vi valgt et språk som C++ som kan kjøre koden mer effektivt og øke hastigheten på beregningene i programmet, ser vi at Python var et godt valg. Vi føler derfor at Python var et bedre valgt enn andre språk som C++ selv med tanke på at C++ har vært det mest brukt programmeringsspråket vi har jobbet med gjennom de tre årene vi har studert på universitetet. Vi ser at selv om vi brukte mye tid på ting som å lære oss språket og ikke jobbe rett på oppgaven vi hadde fremfor oss. Mener vi at Python fortsatt hadde vært mer effektivt å bruke i dette prosjektet.

	
\subsection{Arbeidsfordeling}
Under hele prosjektet har vi jobbet godt sammen, men arbeidsfordelingen kunne vært bedre. Jon Anders hadde en god bakgrunn i matematikk som jeg manglet i mitt studie som programvareutvikler. Dette førte til noen vasker i starten av prosjektet ettersom dette er en matematisk tung oppgave. Vi valgte å løse dette problemet med pair programming som virket bra, spesielt når vi jobbet på et helt nytt språk. Dette gjorde så vi begge lærte språket i samme tempo og endte opp på samme nivå. Mitt mangel på matematisk bakgrunn kom fort frem å gjorde at Jon Anders fikk gjort mer av programmeringen enn meg som måtte bruke mye tid på å forstå de matematiske prinsippene vi brukte. Dette førte til at jeg ikke var like effektiv å fikk gjort mindre av kodingen. Det ble da tyngre og tyngre for meg å holde følge med det som ble gjort jo lengre ut i utviklingsfasen vi kom. Vi innså at dette og valgte at jeg tok mer av annet arbeid som koordinering med Nammo og jobbe med rapportarbeid, når jeg ikke fikk kodet mer.



\subsection{estimering}
Vi hadde startet med en estimering der vi skulle bli ferdig i god tid, i tillegg hadde vi estimert tid for feil og problemer som bug. Den originale estimeringen endte med å være helt feil. Vi var ikke i nærheten av det vi hadde planlagt. Dette kom ikke som en overraskelse, vi forventet at vi ville bomme på estimeringen. Når vi startet bachelor oppgaven hadde vi aldri jobbet med et prosjekt på denne størrelsen og vi hadde kun noen få erfaringer med estimering på små prosjekter. Med vår uerfarenhet og det at vi måtte jobbe grafisk visning og et språk vi aldri hadde sett før, var vi klar over at estimeringen ikke ville være riktig. Vi måtte etterjustere estimeringen ettersom vi kom lengre ut i prosjekt. I den originale estimeringen hadde vi håpet å ha ca en måned til kun rapport arbeid. Vi oppdaget raskt at dette ikke ville være saken og vi ble etterhvert usikker på om vi ville bli ferdig. Når vi hadde kommet godt inn i utviklingsfasen oppdaget vi at vi ikke ville bli ferdig. Vi diskuterte dette med Nammo, så kom vi til en enighet om at vi skulle legge fra oss prosjektet med en godt grunnleggende algoritme som kunne bygges på videre.

\subsection{Hva kunne gjort anderledes}
Når vi ser tilbake på hva som kunne blitt gjort annerledes er det noen ting vi ville endret. Det største hadde vært å satt opp en mer strukturert hverdag. Hadde vi laget over mindre perioder kunne vi ha strukturert ting bedre og kanskje vært mer effektive. Dagene våre hadde ikke en fast starttid og fast arbeidsplass. Dagene vår startet med en fast jakt etter en arbeidsplass der vi kunne jobbe i fred, og diskutere mellom hverandre. på grunn av mangel av grupperom og det at klassen ikke har en av satt klasserom eller arbeidsplass var det ikke lett å finne seg et sted å jobbe. konkurransen på grupperom er stor og en må være tidlig ute og bestille rom 2 uker frem i tid. Hadde vi vært flinkere på å finne en fast arbeidsplass med god plan over hva som måtte gjøres innen denne dagen og denne uken. Istedenfor å jobbe på en fast arbeidsplass med en strukturert plan, ble vi sittende å bare jobbe videre, vi visste hva vi måtte gjøre ettersom det er en oppgaven som består av få kompliserte oppgaver istedenfor et prosjekt som for eksempel utvikling av en app der det er mye forskjellige implantasjoner en på holde styr på.\\ \\
Når vi startet oppgaven møtte vi tidlig på spørsmålet om vi skulle bygge programvaren med eksplisitte- eller implisitte flate. Vi valgte å bruke eksplisitt flate, fordi vi var redd for at en implisitt løsning ikke ville være nøyaktig nok. Vi valgte derfor å bruke eksplisitt flate som hadde større utfordringer og krevde mer arbeid, men hadde større sannsynlighet for mer nøyaktighet. Når vi ser tilbake på hvilke valg vi har tatt og hvordan vi har valgt å løse forskjellige problemer, er implisitt løsning det vi har vurdert mest. Med løsningen vi har når beregner vi en liten del av rakettmotoren, dette er en effektiv løsning som sparer minneforbruket. Hadde vi valgt en implisitt løsning hadde vi ikke møtt på problemet med kollapsende vegger å aldri tenkt på å beregne en liten bit av rakettmotoren. Når vi nå ser tilbake trur vi at en implisitt løsning kunne vært bedre. Noen av løsningene vi har brukt i vår versjon kan bli brukt i en implisitt løsning å senke minneforbruket til et nivå der vi kanskje kan oppnå nøyaktigheten vi trenger. Vi mener da at hadde vi visst det vi gjør nå hadde vi valgt en implisitt løsning som kun beregner en halv stjerne arm og speiler denne. Noen av de største problemene vi har møtt med en eksplisitt løsning er forflytting, detektere når brennstoffet har nådd tankveggen og symmetrilinjen. Disse problemene hadde vært mye mindre jobb og kunne vært mer effektiv.


\subsection{Valg av brukergrensesnitt}
Når vi undersøkte hvordan vi skulle lage brukergrensesnittet valgte vi Tkinter. Dette kom anbefalt fra Python miljøet, i tillegg til at tkinter er kompatibelt med MatPlotLib. Tidlig i utviklingsfasen var vi fornøyd med Tkinter, vi fikk lagt inn lagt inn grafen vi ønsket ved bruk av MatPlotLib. Vi møtte på noen små utfordringer, men fikk hoved funksjonaliteten til å virke som vi ønsket. Når vi senere skulle jobbe med design av brukergrensesnittet, møtte vi på store problemer. Vi hadde brukt en funksjon Tkinter har som heter Pack(). Denne dytter elementet inn på den første plassen som er ledig. Denne plasserte ikke grafen der vi ønsket den og når vi prøvde andre løsninger Tkinter tilbydde hadde vi ingen suksess. Den andre løsningen Tkinter tilbyr er en Grid() funksjon som deler vinduet opp i et grid og lar deg plassere elementene der du ønsker. Denne løsningen virker ikke med MatPlotLib vi hadde derfor elementer vi ikke kunne kontrollere. Vi oppdaget dette på et stadie der vi ikke hadde tid til å gå tilbake å endre på det. qt så ut som en løsning, men visste ikke om den var kompatibel med MatPlotLib. Med det vi vet i dag hadde vi sett på qt først for å se om vi kunne brukt MatPlotLib eller en annet bibliotek for å vise frem elementene våre.





\subsection{Vidreutvikling}
Det er helt klart rom for videreutvikling, vi ble ikke ferdig med oppgaven som vi ønsket. Vi fikk en fungerende 2D modell som hadde en god grunnleggende algoritme, men vi ble ikke ferdig med 3D modellen. 3D modellen var noe vi ønsket vi skulle bli ferdig med, men kom ikke så langt som vi ønsket. Det er her helt klart en utvidelse fra 2D til 3D. I tillegg til en utvidelse til 3D er det små justeringer for å fin pusse algoritmen til å kjøre mer effektivt. Det er noen deler av koden som kanskje er mer effektiv i et annet språk som C++. Det er da mulighet for å ta en funksjon som kan lages i C++, dette er noe som må utforskes videre. 
Etter vi innså at vi ikke ville bli ferdig med den originale oppgaven, Diskuterte vi hva de ville ende opp med når vi var ferdig. Nammo hadde tidligere forsøkt å løse denne oppgaven uten noe suksess. De ønsket derfor at vi skulle gjøre den grunnleggende algoritmen til å virke optimalt på 2D modellen. De hadde planer om å fortsette med prosjektet videre etter vi hadde levert bacheloroppgaven. Et av medlemmene på gruppe har allerede akseptert jobb fra Nammo med stilling for videreutvikling av prosjektet. 





